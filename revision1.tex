\documentclass[12pt]{article}
\usepackage{lmodern} % Pour une police plus nette
\usepackage{stmaryrd}
\usepackage{graphicx} % Pour l'insertion d'images
\usepackage{float}    % Pour contrôler précisément le placement
\usepackage[utf8]{inputenc}
\usepackage[french]{babel}
\usepackage[T1]{fontenc}
\usepackage{hyperref}
\usepackage{verbatim}
\usepackage{color, soul}
\usepackage{pgfplots}
\pgfplotsset{compat=1.18} % Version plus récente de pgfplots
\usepackage{mathrsfs}
\usepackage{amsmath}
\usepackage{amsfonts}
\usepackage{amssymb}
\usepackage{tkz-tab}
%\author{Destiné aux élèves de Terminale S\\Lycée de Dindéfelo\\Présenté par M. BA}
%\title{\textbf{Rappels et compléments sur les fonctions numériques}}
%\date{\today}
\usepackage{tikz}
\usetikzlibrary{arrows, shapes.geometric, fit}
% Commande pour la couleur d'accentuation
\newcommand{\myul}[2][black]{\setulcolor{#1}\ul{#2}\setulcolor{black}}
\newcommand\tab[1][1cm]{\hspace*{#1}}
\usepackage[margin=2.5cm]{geometry} % Ajustement des marges
\usepackage{eso-pic} % Pour ajouter des éléments en arrière-plan

% Commande pour ajouter du texte en arrière-plan, centré au milieu de chaque page
\AddToShipoutPicture{
    \AtPageCenter{%
        \makebox(0,0)[c]{\rotatebox{60}{\textcolor[gray]{0.9}{\fontsize{2cm}{2cm}\selectfont PGB}}}
    }
}

\begin{document}

\noindent
\begin{minipage}[t]{0.48\textwidth}
\raggedright
\textbf{Ministère de l'Éducation Nationale}\\
Inspection Académique de Kédougou\\
Lycée Dindéfelo\\
Cellule de Mathématiques
\end{minipage}
\hfill
\begin{minipage}[t]{0.48\textwidth}
\raggedleft
\textbf{Année scolaire 2024-2025}\\
Date : 08/12/2024\\
Classe : Terminale S2\\
Professeur : M. BA
\end{minipage}

\vspace{1cm}

\begin{center}
\textbf{\underline{Révision : 1}}
\end{center}

\section*{\underline{Problème 1}}
Soit \( f(x) = \frac{x^2 + x + 1}{x - 1} \).
\begin{enumerate}
    \item Déterminer le domaine de définition de \( f \).
    \item Calculer les limites aux bords de \( D_f \).
    \item Préciser les branches infinies.
    \item Déterminer les réels \( a \), \( b \), et \( c \) pour que \( f(x) = ax + b + \frac{c}{x - 1} \). En déduire que la droite \( (D) \) d'équation \( y = x + 2 \) est une asymptote oblique à \( (C_f) \) en \( +\infty \) et \( -\infty \).
    \item Étudier la position de \( (C_f) \) par rapport à \( (D) \).
    \item Montrer que le point \( I(1, 3) \) est centre de symétrie de \( (C_f) \).
    \item Calculer la dérivée de \( f \) et étudier le signe de la dérivée.
    \item Dresser le tableau de variation.
    \item Étudier les intersections de \( (C_f) \) avec les axes de coordonnées.
    \item Tracer \( (C_f) \).
\end{enumerate}

\section*{\underline{Problème 2}}

\[\text{Soit }
f(x) =
\begin{cases} 
    -x + 2 - \frac{2x}{1 + x^2}, & \text{si } x \leq 1, \\
    \frac{1 - x}{\sqrt{x^2 - 2x + 5}}, & \text{si } x > 1.
\end{cases}
\]

\begin{enumerate}
    \item Déterminer \( D_f \).
    \item Calculer les limites aux bords.
    \item Étudier les branches infinies.
    \item Étudier la continuité de \( f \) en 1.
    \item Étudier la dérivabilité de \( f \) en 1. Interpréter les résultats.
    \item Montrer que :
    \begin{enumerate}
        \item \( \forall x \in ]-\infty, 1[, f'(x) = \frac{-x^4 + 3}{(1 + x^2)^2} \) et préciser son signe.
        \item \( \forall x \in ]1, +\infty[, f'(x) = \frac{-4}{\sqrt{x^2 - 2x + 5} \, (\sqrt{x^2 - 2x + 5})^2} \).
    \end{enumerate}
    \item Dresser le tableau de variation.
    \item Tracer \( (C_f) \).
\end{enumerate}


\section*{\underline{Problème 3}}

\subsection*{Partie A :}

Soit \( f \) la fonction définie par :
\[
f(x) = x - 1 - \sqrt{x^2 - x}.
\]

\begin{enumerate}
    \item Déterminer \( D_f \).
    \begin{enumerate}
        \item Calculer \( \lim_{x \to -\infty} f(x) \) et \( \lim_{x \to +\infty} f(x) \).
        \item Étudier la branche infinie de la courbe \( (C_f) \) au voisinage de \( -\infty \).
        \item Étudier la branche infinie de la courbe \( (C_f) \) au voisinage de \( +\infty \).
    \end{enumerate}

    \item Étudier la dérivabilité de la fonction \( f \) à droite de 1 et à gauche de 0, puis interpréter géométriquement les résultats obtenus.
    \begin{enumerate}
        \item Justifier la dérivabilité de la fonction \( f \) sur \( ]-\infty, 0[ \cup ]1, +\infty[ \), puis montrer que pour tout \( x \) de \( ]-\infty, 0[ \cup ]1, +\infty[ \) :
        \[
        f'(x) = \frac{2\sqrt{x^2 - x} - (2x - 1)}{2\sqrt{x^2 - x}}.
        \]
        \item Montrer que :
        \[
        (\forall x \in ]-\infty, 0[), f'(x) > 0 \quad \text{et} \quad (\forall x \in ]1, +\infty[), f'(x) < 0.
        \]
        \item Dresser le tableau de variations de la fonction \( f \).
    \end{enumerate}

    \item Tracer la courbe \( (C_f) \) dans un repère orthonormé \( (O, \vec{i}, \vec{j}) \).
\end{enumerate}
\subsection*{Partie B :}

On considère la fonction \( g \), la restriction de la fonction \( f \) sur \([2, +\infty[\) :
\[
g(x) = f(x), \quad x \geq 2.
\]

\begin{enumerate}
    \item Montrer que \( g \) admet une fonction réciproque \( g^{-1} \) définie sur un intervalle \( J \) qu'on déterminera.
    \item Calculer : \( g^{-1}(2 - 2\sqrt{2}) \). (On donne : \( g(4) = 2 - 2\sqrt{2} \)).
    \item Déterminer \( g^{-1}(x) \) pour tout \( x \in J \).
    \item Tracer la courbe \( (C_{g^{-1}}) \) dans le même repère orthonormé \( (O, \vec{i}, \vec{j}) \).
\end{enumerate}

\end{document}
